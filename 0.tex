%#!platex --src-specials main.tex

\Title{触覚情報における加速度情報の取り扱いに関する検討} 
\Author{塩田 雅人}
\Date{2}{3}{5}

%学部の人は次のコメント行の%を外してください.
\senior  %この\seniorは\Synopsisの前であればどこに書いてもかまいません.

\Synopsis

\begin{Abstract}
%アブスト

近年,バーチャルリアリティ技術の発展とともに,ヘッドマウントディスプレイなどリッチな視聴覚情報が利用可能になりつつある.それにあわせ,触覚情報の実現も求められている.そのため,さまざまな触覚情報を提示する触覚ディスプレイが数多く研究されている.このような触覚ディスプレイのためのコンテンツとして,センサを用いて取得した触覚情報の分類が研究されつつある. その中でも加速度触覚情報は再現性に優れ, 触覚提示の研究へ発展が期待される. 
現在, 3軸加速度触覚情報の分類を行う際, 3軸を1軸の振動情報に統合し, 情報量を抑える手法がある. 
しかし3軸加速度情報から 1 軸の振動情報に次元削減を行う中で, 欠損した情報も再現性を考えると触覚情報において重要である可能性がある. 
本研究では,次元削減と正規化を行った3軸加速度触覚情報と, 無加工の 3 軸加速度触覚情報とをそれぞれ畳み込みニューラルネットワークを用いた機械学習で分類, 比較し,  分類精度の観点から触覚情報における 3 軸加速度情報の取扱いについて検討する. 
\end{Abstract}

%
% これより下は学部(卒業論文を書く人)には関係ありません. 
% 
% \title{English Titile}
% \author{Taro Kumamoto}
% \endate{February}{Day}{Year}

% \synopsis

% \begin{abstract}
% English
% \end{abstract}










% 修士論文の論文概要

% 修士論文については和文と英文の論文概要を次の要領で作成し,【5】のb.2として下
% さい。英文で本文を記述した場合も,論文概要は和文,英文の両方で作成することが望ま
% しい(詳細は指導教員の指示に従って下さい)。

% 1.論文概要(和文)の形式

%     a.修士論文と同じ体裁で作った表紙(図2)を付けます。ただし,題目と氏名の間
%     に,「論文概要」と書き添えて下さい。

%     b.研究の目的,論文全体のあらまし,各章の内容(簡単に),結論(やや詳し
%     く),得られた成果の意義を,この順序で3~5ページ程度にまとめて下さい。

% 2.論文概要(英文)の形式

%    a.修士論文と同じ体裁で作った表紙(図2)を英文で記載して付けます。ただし,題
%    目と氏名の間に, Synopsis と書き添えて下さい

%    b.研究の目的,論文全体のあらまし,結論を,100~300 語程度にまとめて下さい。


% 3.和文,英文ともに目次は付けないで下さい。また,原則として,図,表,式を用いな
% いで下さい。
