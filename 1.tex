%#!platex --src-specials main.tex

\chapter{はじめに}
\pagenumbering{arabic}     % 絶対必要です.最初の章にのみ記述します.
%プレ 羅列のみ
\section{研究背景と目的}
モバイルデバイスの発達により我々はこれまでより多くの情報を取り入れるようになった. それに伴い情報と人が接する面であるインタフェースも重要性を増し, より人の感覚に沿うようなインタフェースが開発されている. 触覚情報を用いたインタフェースはヒトが知覚する触感覚の精密さとヒトの触感覚受容器官の自由度から注目されてきているが, 受容される触覚情報に基づいた触覚刺激を提示する, 触覚ディスプレイに関連する研究は研究分野としてはまだ新しく, 視覚における光学ディスプレイの情報提示手法に比べ確立に至っていない. そのため現在ヒトに受容される触覚情報を収集する研究が様々な手法で行われている \cite{abdulali2016data},\cite{strese2017content}.
手法により収集する触覚情報は異なっているが, そのなかでも触覚情報を加速度情報として収集する手法は再現性が高く, 多数の関連研究が存在する
\cite{kuchenbecker2010verrotouch},\cite{romano2011creating}. 
またこれらの研究の多くは触覚ディスプレイを用いた情報の適切な提示が可能となるように, 触覚情報の収集手法に加え, 得られた触覚情報の分類手法も検討している. 触覚情報の分類は主に機械学習によって行われており, 分類の精度向上が多くの研究の課題となっている. 

3軸加速度の触覚情報の取扱いについて, 多くの研究が用いている手法の一つに, 3軸加速度情報を1軸の振動情報に統合し, 情報量を抑える手法がある. これは皮膚構造中の振動刺激を知覚する受容器が振動方向の判別を得意としていないこと\cite{brisben1999detection}に加え触覚情報に基づいたテクスチャデータをレンダリングする際に, 計算コストを抑える事が可能であることに起因する. 
しかし3軸加速度情報から1軸の振動情報に次元削減を行う中で, 欠損した情報も再現性を考えると触覚情報において重要である可能性がある \cite{kurogi}.

本研究では, Agatsumaら\cite{agatsuma}によって得られた3軸加速度情報をもとに,次元削減などの信号処理をおこなった情報をそれぞれ畳み込みニューラルネットワーク (Convolutional Neural Network:  CNN \cite{lecun}) を用いた機械学習で分類し, 分類精度の観点から触覚情報における3軸加速度情報の取扱いについて検討する. 

本稿において触覚情報とはヒトの受容器官と触対象との間に生じた摩擦に起因する振動, 受容器によって受容される圧力, 熱等の情報をいう. 

\section{本論文の構成}
本章では, 本研究の背景と目的を述べた. 本稿の次章以降の構成を以下に示す. \ref{chap:haptic}章では, 本研究における前提知識と関連する先行研究を述べる. \ref{chap:method}章では, 本研究で用いる触覚情報の次元削減手法と機械学習手法について述べる. \ref{chap:exp}章では, 本研究の評価実験について述べる. \ref{chap:result}章では, 実施した評価実験に対しての得られた結果を掲載する. \ref{chap:cons}章では, 得られた結果に対しての考察を行う. \ref{chap:conc}章で結論を述べる.

% Local Variables:
% TeX-master: "main"
% mode: yatex
% End:











